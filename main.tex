\documentclass[turkish]{article}
\usepackage{graphicx}
\usepackage[T1]{fontenc}
\usepackage[utf8]{inputenc}
\usepackage{lmodern}
\usepackage{hyperref}
\usepackage{caption}
\usepackage{times}
\usepackage{amsmath}
\usepackage{lipsum}
\usepackage[margin=1.3in]{geometry}
\usepackage{longtable}
\usepackage{pdflscape}
\usepackage{lscape}
\usepackage{tabularx}
\usepackage{url}
\usepackage{hyperref}


\renewcommand{\contentsname}{İçindekiler}
\title{Anket Düzenleme Dönem Ödevi}
\author{Pınar Beydili}
\date{May 2024}

\begin{document}
\begin{titlepage}
    \begin{center}
 \begin{center}
            
            \centering
            \includegraphics[width=0.5\linewidth]{Imgs/hacettepelogo.jpg}
            \caption{}
            \label{}
        \end{center}   
        
        \vspace*{1cm}
        \Huge
        \textbf{Hacettepe Üniversitesi Beytepe Kampüsünde Bulunan Fakültelerde Eğitim Gören Ögrencilerin HPV Hakkında
Bilgi Tutum ve Davranışlarının İncelenmesi }
 
        \vfill
   
        \Large
        \textit{Derya Ahsen Sarıçiçek ~2200329076 \\
        Pınar Beydili~2200329807\\
        Sena Şükran Şahin~2200329011 \\
        }\\
        
        \vfill
        Hacettepe Üniversitesi \\
        İstatistik \\
        
    \end{center}
\end{titlepage}

\newpage

\tableofcontents
\clearpage

\section{MATERYAL VE YÖNTEM}

\subsection{Araştırmanın Amacı}
Bu araştırma üniversite öğrencilerinin “HPV Hakkında Bilgi, Tutum ve Davranışlar”ının incelenmesi amacıyla Hacettepe Üniversitesi Beytepe Kampüsünde yapılmıştır. 
Araştırma kapsamına 2023-2024 öğretim yılında Hacettepe Üniversitesi Beytepe Kampüsünde eğitim gören öğrenciler alınmıştır. Bilgiler anket yöntemi- çevrimiçi anket yoluyla toplanmıştır. Katılımcılara kişisel bilgileri dışında virüs,test ve aşı bilgileri içeren  sorular sorulmuştur. Elde edilen verilere göre anketin değerlendirilmesi yapılmıştır. 

\subsection{Kullanılan Örnekleme Yöntemi}
Bu çalışmada kullanılan örnekleme yöntemi Rastgele Örnekleme'dir. Bu yöntemde kampüs içerisindeki tüm bireyler eşit şansa sahip olacaktır. Örneklem büyüklüğü tablo yardımıyla, \%95 güven düzeyi ve \%5 hata payı ile 381 olarak belirlenmiştir.   

   \begin{center}
     \centering
    \includegraphics[width=0.75\linewidth]{Imgs/image.png}
    \caption{}
    \label{}
 \end{center}

 
\subsection{Anket Soru Formunun Hazırlanması}
Araştırmada uygulanan soru formu, araştırmada görev yapan Hacettepe Üniversitesi İstatistik Bölümü öğrencileri tarafından hazırlanmış ve değerlendirilmiştir. Soru formuna Hacettepe Üniversitesi Fen Fakültesi İstatistik bölümünde okumakta olan öğrencilere yapılan ön çalışma sonucunda son şekli verilmiştir. 

\subsection{İstatistiksel Analizler}
Çevrimiçi anket yöntemiyle elde edilen veriler SPSS programı ile analiz edilmiştir. Analizler öncesinde verilerin normalliği Shapiro Wilk'in normallik testi ve Q-Q grafikleri yardımıyla,grup varyanslarının homojenliği ise Levene testi ile kontrol edilmiştir.Çalışmada yer alan kategorik değişkenler için sıklık(n) ve yüzdelik(\%) olarak sunulmuştur.Değişkenler arasındaki ilişkinin kontrolü için çapraz tablo çözümlemesi uygulanmıştır. Değişkenler arasındaki farklılıkların değerlendirilmesi için ise Kruskal-Wallis ve Mann-Whitney testleri kullanılmıştır.

\subsection{HPV Nedir?}
Human Papilloma Virüsü (HPV), yaklaşık 200 farklı tipi olan ve çeşitli kanser türlerine yol açabilen çift sarmallı bir DNA virüsüdür. HPV tipleri, kanser risklerine göre düşük, orta ve yüksek riskli olarak sınıflandırılır.
\\Düşük Riskli HPV Tipleri: 6, 11, 40, 42, 43, 44, 54, 61, 70, 72, 81,
\\Orta Riskli HPV Tipleri: 26, 53, 66,
\\Yüksek Riskli HPV Tipleri: 16, 18, 45, 31, 33, 52, 58, 35, 59, 56, 51, 39, 68, 73, 82’dir.
\\En az 40 HPV tipi siğillere neden olmaktadır. Çoğu HPV enfeksiyonu, herhangi bir belirti göstermeden geçici ve akut bir süreçtir. Bu nedenle, birçok kişi enfekte olduğunun farkında olmayabilir. Enfeksiyonların yaklaşık \%80’i bağışıklık sistemi tarafından kontrol altına alınırken, \%5-10’luk bir kısmı kalıcı hale gelebilir ve zamanla kansere dönüşebilir.
\\HPV enfeksiyonları, hem erkeklerde hem de kadınlarda üreme organları, anüs çevresi ve yutak bölgesinde kansere neden olmaktadır. HPV, rahim ağzı kanserlerinin \%99’u, anal kanserin \%90’ı, vajinal kanserlerin \%65’i, vulvar kanserlerin \%50’si, penis kanserlerinin \%60’ı ve orofaringeal kanserlerin \%60-70’i için sorumludur. Her yıl dünya çapında yaklaşık 690.000 yeni HPV ilişkili kanser vakası tanı konmaktadır.
\\HPV, cilt teması, ciltteki yaralar, mikroplu yüzeyler ve doğum yoluyla geçebilir; bu bulaşma doğrudan temasla veya dolaylı yollarla gerçekleşebilir. Ancak, en yaygın bulaşma yolu cinsel ilişkidir. Cinsel olarak aktif bireyler ve güvensiz cinsel ilişki yaşayanlar, özellikle risk altındadır. ABD’de yapılan bir çalışmada, ilk cinsel deneyimden sonra kadınların \%39’unda HPV enfeksiyonu geliştiği bildirilmiştir.
\\HPV enfeksiyonu risk faktörleri arasında birden fazla cinsel eş, güvensiz cinsel ilişki, erken yaşta cinsel ilişki (<16 yaş), aşılanmama, yetersiz tarama programları, sigara kullanımı, bağışıklık sistemini zayıflatan durumlar, kötü hijyen koşulları ve düşük sosyoekonomik düzey bulunmaktadır.

\subsection{HPV'den Nasıl Korunulur?}
HPV enfeksiyonundan korunmada, HPV aşıları büyük önem taşımaktadır. Bu aşılar, özellikle kansere ve siğillere neden olan HPV tiplerine karşı geliştirilmiştir. Mevcut ikili, dörtlü ve dokuzlu HPV aşıları bulunmaktadır. İkili ve dörtlü aşılar, rahim ağzı kanserlerinin \%70-80’inden sorumlu olan HPV 16 ve 18 tiplerine karşı koruma sağlarken, dokuzlu aşı, kanserlerin \%88-90’ından sorumlu olan HPV 16, 18, 31, 33, 45, 52 ve 58 tiplerine karşı \%100 koruma sunar. Ayrıca, dörtlü ve dokuzlu aşılar, genital siğillerin \%90 nedeni olan HPV 6 ve 11’e karşı \%100 ek koruma sağlar.
\\Kuzey Avrupa’daki aşı programları, aşılanan bireylerde uzun süreli koruma sağladığını göstermiştir. DSÖ, EMA, FDA ve Türkiye Cumhuriyeti Sağlık Bakanlığı, ilk cinsel temastan önce 9-26 yaş arasındaki her iki cinsiyete de HPV aşısı yapılmasını tavsiye etmektedir. Ancak, Türkiye’de HPV aşısı henüz ulusal aşı programına dahil edilmemiştir. Araştırmalar, Türkiye’deki ebeveynlerin çoğunun HPV enfeksiyonu ve aşısı hakkında yeterli bilgiye sahip olmadığını göstermektedir. Bu durum, aşılanma oranlarının düşük olmasına neden olmaktadır. Dünya genelinde, HPV aşıları 87 ülkede ulusal aşı programlarına dahil edilmiş ve gelişmiş ülkelerde okul temelli aşı programları uygulanmaktadır. ANA, gençler arasında aşılanma oranlarını artırmak için toplum temelli çalışmaların önemini vurgulamaktadır.
\\HPV’den korunmada birincil yöntem aşılanma olup, ikincil koruma ise HPV-DNA testi ve Papanicolaou testi (Pap-smear) ile yapılan tarama programları ile sağlanmaktadır. Aşılar tüm HPV tiplerine karşı koruma sağlamadığı için, düzenli tarama yaptırmak önemlidir. Serviks kanseri, dünya genelinde kadınlarda en çok görülen dördüncü kanser türüdür. Türkiye’de ise, 15-44 yaş arasında en sık görülen ikinci kanser türü serviks kanseridir. Sağlık Bakanlığı, 30-65 yaş grubundaki her kadının beş yılda bir HPV-DNA testi ile taranmasını ve pozitif çıkan vakaların pap-smear testi ile tekrar değerlendirilmesini önermektedir.
 
\hspace{}
\hspace{}
\hspace{}

\section{GRAFİKLER}


    \begin{figure}[ht]
    \centering
    \includegraphics[width=0.5\textwidth]{Imgs/cinsiyet.png}
    \caption{}
    \label{cinsiyet}
\end{figure}  
Resim \ref{cinsiyet}'den hareketle, çalışmaya katılan Hacettepe üniversitesi Beytepe kampüsünde eğitim gören 374 öğrencinin 176’sı (\%47,1) erkek , 198’i (\%52,9) kadındır.
\\

   \begin{figure}[ht]
    \centering
    \includegraphics[width=0.7\textwidth]{Imgs/fakülte.png}
    \caption{}
    \label{fakülte}
\end{figure}     
Resim \ref{fakülte}'den hareketle, öğrencilerin  2’si(\%0,5) Ankara Devlet Konservatuvarı , 42’si (\%11,2) Edebiyat Fakültesi ,35’i(\%9,4) Eğitim Fakültesi , 89’u (\%23,8) Fen Fakültesi , 14’ü (\%3,7) Güzel Sanatlar Fakültesi , 37’si (\%9,9) Hukuk Fakültesi ,44’ü(\%11,8) İktisadi ve İdari Bilimler Fakültesi ,18’i (\%4,8) İletişim Fakültesi ,55’i(\%14,7) Mühendislik Fakültesi ,10’ u (\%2,7) Spor Bilimleri Fakültesi ,28’i (\%7,5) ise diğer fakültelerden katılım göstermiştir.
\clearpage
    \begin{figure}[ht]
    \centering
    \includegraphics[width=0.4\textwidth]{Imgs/ikamet.png}
    \caption{}
    \label{ikamet}
\end{figure}   
Resim \ref{ikamet}'ten hareketle eğitim döneminde ikamet edilen yerlerden aile evi \%32’sini (122 kişi) ,devlet yurdu \%26’sını (100 kişi),öğrenci evi \%30,7’sini (115 kişi) ,özel yurt ise \%9,9’unu (37 kişi) oluşturmaktadır.

      \begin{figure}[ht]
    \centering
    \includegraphics[width=0.5\textwidth]{Imgs/bölge.png}
    \caption{}
    \label{bölge}
\end{figure}   
Resim \ref{bölge}'ten hareketle, katılımcıların hayatlarının en uzun dönemini geçirdikleri bölge, Akdeniz Bölgesi olanlar \%15.2’sini (57 kişi), Doğu Anadolu Bölgesi olanlar \%8’ini (30 kişi), Ege Bölgesi olanlar \%12,6’sını (47 kişi), Güneydoğu Anadolu Bölgesi olanlar \%7’sini (26 kişi), İç Anadolu Bölgesi olanlar \%23,8’ini (89 kişi), Karadeniz Bölgesi olanlar \%8,6’sını (32 kişi) ve Marmara Bölgesi olanlar ise \%24,9’unu (93 kişi) oluşturmaktadır.

   \begin{figure}[ht]
    \centering
    \begin{minipage}[b]{0.4\textwidth}
        \centering
        \includegraphics[width=\textwidth]{Imgs/anne.png}
        \caption{}
        \label{anne}
    \end{minipage}
    \hfill
    \begin{minipage}[b]{0.45\textwidth}
        \centering
        \includegraphics[width=\textwidth]{Imgs/baba.png}
        \caption{}
        \label{baba}
    \end{minipage}
\end{figure}
Resim \ref{anne}'ten hareketle, anne eğitim durumu okuryazar olmayanlar 10 kişi (\%2,7), ilkokul olanlar 62 kişi (\%16,6), ortaokul olanlar 66 kişi (\%17,6), lise olanlar 101 kişi (\%27), üniversite olanlar 120 kişi (\%32,1) ve yüksek lisans/doktora olanlar ise 15 kişidir (\%4).Resim \ref{baba}'dan hareketle,baba eğitim durumu okuryazar olmayanlar 3 kişi (\%0,8), ilkokul olanlar 35 kişi (\%9,4), ortaokul olanlar 46 kişi (\%12,3), lise olanlar 114 kişi (\%30,5), üniversite olanlar 145 kişi (\%38,8) ve yüksek lisans/doktora olanlar ise 31 kişidir (\%8,3). 
\clearpage
      \begin{figure}[ht]
    \centering
    \includegraphics[width=0.5\textwidth]{Imgs/gelir.png}
    \caption{}
    \label{gelir}
\end{figure}   
Resim \ref{gelir}'den hareketle, aile aylık geliri 17.000 TL altı olanlar 10 kişi (2,7), 17.000-34.000 TL olanlar 70 kişi (\%18,7), 34.001-61.000 TL olanlar 118 kişi (\%31,6), 61.001-100.000 TL olanlar 105 kişi (\%28,1), 100.001-130.000 TL olanlar 43 kişi (\%11,5) ve 130.000 TL üstü olanlar ise 28 kişidir (\%7,5).

\begin{figure}[ht]
    \centering
    \includegraphics[width=0.5\textwidth]{Imgs/sigorta.png}
    \caption{}
    \label{sigorta}
\end{figure}   
Resim \ref{sigorta}'den hareketle,Sağlık sigortasına sahip olmayanlar 18 kişi (\%4,8), devlet sağlık sigortasına sahip olanlar 241 kişi (\%64,4), özel sağlık sigortasına sahip olanlar 66 kişi (\%17,6), hem devlet hem özel sağlık sigortasına sahip olanlar ise  49 kişi (\%13,1) olduğu görülmektedir.

\begin{figure}[ht]
    \centering
    \includegraphics[width=0.5\textwidth]{Imgs/aktiflik.png}
    \caption{}
    \label{aktiflik}
\end{figure}  
Resim \ref{aktiflik}'dan hareketle,160 kişi (\%42,8) cinsel olarak aktif,158 kişi (\%42,2) cinsel olarak aktif değil , 56 kişi (\%15) ise belirtmek istememektedir.
\clearpage
    
   \begin{figure}[ht]
    \centering
    \begin{minipage}[b]{0.45\textwidth}
        \centering
        \includegraphics[width=\textwidth]{Imgs/duyum1.png}
        \caption{}
        \label{duyum1}
    \end{minipage}
    \hfill
    \begin{minipage}[b]{0.45\textwidth}
        \centering
        \includegraphics[width=\textwidth]{Imgs/duyum1.1.png}
        \caption{}
        \label{duyum1.1}
    \end{minipage}
\end{figure}
Resim \ref{duyum1}'dan hareketle, katılımcıların \%89,8’i (336 kişi) HPV’yi daha önce duymuş ,\%10,2’si (38 kişi) ise  HPV ‘yi daha önce duymamıştır.
\\ Resim \ref{duyum1.1}'den hareketle\textit{"HPV'yi duyduysanız nereden duydunuz?"} sorusuna verilen yanıtlara göre,\%9,2’si (33 kişi) aileden , \%8,1’i (29 kişi) bilimsel makale/araştırma dergilerinden ,\%17,6’sı (63 kişi) eğitim kurumlarından(dersler,seminerler vs) , \%17,6’sı (63 kişi) medyadan (televizyon,gazete,dergi ), \%9,7’si (35 kişi) sağlık profesyonellerinden (doktor,hemşire,sağlık danışmanı vs) ,\%17,6’sı (63 kişi) sosyal çevreden ,\%17,9 ‘u (64 kişi) sosyal medyadan , \%2,2’si ise HPV’yi diğer yollardan duymuştur.

   \begin{figure}[ht]
    \centering
    \includegraphics[width=0.4\textwidth]{Imgs/test.png}
    \caption{}
    \label{test}
\end{figure}   
Resim \ref{test}'den hareketle, daha önce HPV testini duyanlar 291 kişi(\%77,8) , duymayanlar ise 83 kişidir(\%22,2).

     \begin{figure}[ht]
    \centering
    \includegraphics[width=0.4\textwidth]{Imgs/asıduyum.png}
    \caption{}
    \label{asıduyum}
\end{figure}   
Resim \ref{asıduyum}'ten hareketle,daha önce HPV aşısını duyanlar 298 kişi (\%79,8) , duymayanlar ise 76 kişidir (\%20,3).


\clearpage
  \begin{figure}[ht]
    \centering
    \includegraphics[width=0.4\textwidth]{Imgs/asiolma.png}
    \caption{}
    \label{asiolma}
\end{figure} 
Resim \ref{asiolma}'ten hareketle,daha önce HPV aşısı olanlar 77 kişi (\%20,6) , olmayanlar ise 297 kişidir (\%79,4).


    \begin{figure}[htbp]
        \centering
        \includegraphics[width=0.4\textwidth]{Imgs/asiolma1.png}
        \caption{}
        \label{asiolma1}
    \end{figure}
Resim \ref{asiolma1}'ten hareketle, daha önce HPV aşısı yaptırmayanlar yaptırmama nedenleri olarak ; 63 kişi (\%19) aşı hakkında yeterli bilgiye sahip olmadıklarını ,128 kişi (\%38,7) aşının maliyetinin yüksek olduğunu , 29 kişi (\%8,8) aşının etkinliği ve güvenilirliği hakkında şüpheleri  olduğunu , 34 kişi (\%10,3) aşı hakkındaki olumsuz görüşlerin kararlarını etkilediğini , 77 kişi (\%23,2) ise kendilerini risk altında hissetmediklerini söylemiştir.

    \begin{figure}[htbp]
        \centering
        \includegraphics[width=0.5\textwidth]{Imgs/asiolmadusunce.png}
        \caption{}
        \label{asiolmadusunce}
    \end{figure}
Resim \ref{asiolmadusunce}'dan hareketle,gelecekte HPV aşısı olmayı düşünenlerin sayısı 195 (\%52,1) , düşünmeyenlerin sayısı 135 (\%36,1) ,halihazırda HPV aşısı olanların sayısı ise 44 ‘tür (\%11,8).
\newpage
\section{BULGULAR}
     
Kadın öğrencilerin HPV’yi  ( \%91,9 ‘a karşı  \%86,4 , p=0,036) , HPV testini  ( \%86,4’e karşı  \%68,2 , p<0,001) ve HPV aşısını ( \%8,4’e karşı  \%72,2 , p=0,001) duyma oranının erkeklere kıyasla anlamlı şekilde yüksek olduğu belirlenmiştir.

Ankete katılan öğrencilerden gelecekte HPV aşısı olmayı düşünen ( \%62,6’ya karşı  \%40,3 , p<0,001) kadınların oranı erkeklere kıyasla anlamlı bir şekilde yüksektir. HPV aşısı olmayı düşünmeyenler( \%46’ya karşı  \%27,3 , p<0,001) ve HPV aşısı olmuş olanlarda(  \%13,6’ya karşı  \%10,1 , p<0,001) erkeklerin oranı kadınlara kıyasla anlamlı bir şekilde daha yüksektir.
‘HPV aşısı oldunuz mu’ sorusuna evet cevabını verenler ile ‘ Gelecekte HPV aşısı olmayı düşünüyor musunuz’sorusuna HPV aşısı oldum cevabını verenlerin sayısında gözlenen farklılığın ;cevaplayanların gelecekte olma sorusuna 2.veya 3.dozu olup olmama olarak yanıtlanmasından kaynaklandığı düşünülebilir.

Ebeveynlerin okuryazar olmadığı tek bir vakada HPV hakkında bilgi sahibi olduğu bildirilmiştir. İlkokul mezunu ebeveynlere sahip bireylerin büyük çoğunluğu ( \%95,2) HPV hakkında bilgi sahibi iken, sadece küçük bir kısmı ( \%4,8) bu konuda bilgisizdir. Ortaokul mezunu ebeveynlere sahip bireylerde de benzer bir eğilim gözlenmekte olup,  \%84,6’sı bilgi sahibi,  \%15,4’ü ise bilgisizdir. Lise mezunu ebeveynlere sahip bireylerin  \%96,1’i HPV hakkında bilgi sahibiyken,  \%3,9’u bilgisizdir. Üniversite mezunu ebeveynlere sahip bireylerde ise  \%96,4’ü bilgi sahibi,  \%3,6’sı bilgisizdir. Yüksek lisans veya doktora mezunu ebeveynlere sahip bireylerde ise  \%90’ı bilgi sahibi,  \%10’u bilgisizdir.

İlkokul mezunu ebeveynlere sahip bireylerin  \%95,2’si HPV testini duymuş,  \%4,8’i duymamıştır. Ortaokul mezunu ebeveynlere sahip bireylerin  \%69,2’si duymuş,  \%30,8’i duymamıştır. Lise mezunu ebeveynlere sahip bireylerin  \%88,2’si duymuş,  \%11,8’i duymamıştır. Üniversite mezunu ebeveynlere sahip bireylerin  \%83,1’i duymuş,  \%16,9’u duymamıştır. Yüksek lisans veya doktora mezunu ebeveynlere sahip bireylerin  \%70’i duymuş,  \%30’u duymamıştır.
İlkokul mezunu ebeveynlere sahip bireylerin  \%81’i HPV aşısını duymuş,  \%19’u duymamıştır. Ortaokul mezunu ebeveynlere sahip bireylerin  \%46,2’si duymuş,  \%53,8’i duymamıştır. Lise mezunu ebeveynlere sahip bireylerin  \%88,2’si duymuş,  \%11,8’i duymamıştır. Üniversite mezunu ebeveynlere sahip bireylerin  \%92,8’i duymuş,  \%7,2’si duymamıştır. Yüksek lisans veya doktora mezunu ebeveynlere sahip bireylerin  \%60’ı duymuş,  \%40’ı duymamıştır.

İlkokul mezunu ebeveynlere sahip tüm bireyler ( \%100) HPV aşısı olmamıştır. Ortaokul mezunu ebeveynlere sahip bireylerin  \%23,1’i aşı olmuş,  \%76,9’u olmamıştır. Lise mezunu ebeveynlere sahip bireylerin  \%15,7’i aşı olmuş,  \%84,3’ü olmamıştır. Üniversite mezunu ebeveynlere sahip bireylerin  \%24,1’i aşı olmuş,  \%75,9’u olmamıştır. Yüksek lisans veya doktora mezunu ebeveynlere sahip bireylerin yarısı ( \%50) aşı olmuş, diğer yarısı ( \%50) olmamıştır.

Ebeveynlerin okuryazar olmadığı tek bir vakada, gelecekte aşı olmayı düşünmeyen bir kişi bulunmaktadır. İlkokul mezunu ebeveynlere sahip 21 kişiden 12’si ( \%57,1) gelecekte aşı olmayı düşündüğünü, 8’i ( \%38,1) düşünmediğini ve 1’i ( \%4,8) zaten aşı olduğunu belirtmiştir. Ortaokul mezunu ebeveynlere sahip 13 kişiden 4’ü ( \%30,8) gelecekte aşı olmayı düşündüğünü, 7’si ( \%53,8) düşünmediğini ve 2’si ( \%15,4) zaten aşı olduğunu söylemiştir. Lise mezunu ebeveynlere sahip 51 kişiden 27’si ( \%52,9) gelecekte aşı olmayı düşündüğünü, 18’i ( \%35,3) düşünmediğini ve 6’sı ( \%11,8) zaten aşı olduğunu söylemiştir. Üniversite mezunu ebeveynlere sahip 83 kişiden 50’si ( \%60,2) gelecekte aşı olmayı düşündüğünü, 21’i ( \%25,3) düşünmediğini ve 12’si ( \%14,5) zaten aşı olduğunu belirtmiştir. Yüksek lisans veya doktora mezunu ebeveynlere sahip 10 kişiden 3’ü ( \%30) gelecekte aşı olmayı düşündüğünü, 4’ü ( \%40) düşünmediğini ve 3’ü ( \%30) zaten aşı olduğunu belirtmiştir.

Öğrencilerin HPV Bilgi Ölçeği ve alt boyutlarına ilişkin toplam puanlara ilişkin dağılımlar Tablo 1’de verilmiştir. Mann-Whitney U testi sonuçları, kadınların HPV ile ilgili bilgi ve tutumlarının erkeklerden daha yüksek olduğunu göstermiştir (p<0,05). Özellikle 1.alt boyut (8,94±0,47’ye karşı 8,15±0,54) ve 3. alt boyutlarda (2,55±0,21’e karşı 2,21±0,21) bu fark daha belirgindir. Kadınların HPV genel bilgi ve HPV aşı testi ile ilgili bilgi ve tutumları erkeklere göre daha yüksek olduğu gözlemlenmiştir.

Katılımcıları hayatlarını en uzun geçirdiği bölge durumuna göre Kruskal Wallis test sonuçları incelediğinde HPV ile ilgili bilgi ve tutumlarının 1. Ve 4. Alt boyut olan HPV genel bilgi ve HPV tarama testleri bilgisi bakımından istatistiksel olarak farklılık gösterdiği saptanmıştır. (p <0.05) Post-Hoc testleri bölge karşılaştırılması sonucu Güneydoğu Anadolu ve Karadeniz Bölgesi’nin diğer bölgelerden istatistiksel olarak farklılık gösterdiği gözlemlenmektedir. HPV genel bilgi (Karadeniz bölgesi için 7,21±1,24, Güneydoğu Anadolu bilgesi için 7,15±0,95) ve HPV tarama testi ile ilgili bilgi tutumları (Karadeniz bölgesi için 2,59±0,65, Güneydoğu Anadolu bilgesi için 2,34±0,50) diğer bölgelere göre daha düşüktür.

Baba eğitim durumu için 2.alt boyutun farklılık gösterdiği gözlemlenmektedir. Bu farklılığa sebep olan durumlar ise ilkokul ve orta okul düzeyi olduğu saptanmıştır. Diğer eğitim seviyelerine göre HPV aşı programı bilgisi ile ilgili bilgi tutumları daha düşük olduğu gözlemlenmiştir.

Aile gelir durumu 1., 2., 4., alt boyutlar olan HPV genel bilgi, HPV aşı programı bilgisi, HPV tarama testi ve toplam skorda istatistiksel olarak farklılık gözlemlenmektedir. Bu farklılıkların 17.000 TL – 34.000TL ve 34.000 TL-61.000 TL gruplarına ait olduğu saptanmıştır. Bu aile gelir gruplarının HPV genel bilgi, HPV aşı programı bilgisi, HPV tarama testi ile ilgili bilgi ve tutumları düşük çıkmıştır.

Sağlık sigortası durumu bakımından 1. Alt boyut olan HPV genel bilgisi için gruplar arası farklılık gözlemlenmiştir. (p<0,05) Farklılığa sebep olan grubun sağlık sigortası olmayanlar olduğu ve diğer gruplara göre HPV genel bilgisi ortalamasının (7,00±1,68) düşük olduğu saptanmıştır.

Cinsel aktivite bakımından 1. Alt boyut olan HPV bilgisi, 3. Alt boyut olan HPV aşı bilgisi ve toplam skor gruplarında istatistiksel olarak anlamlı bir farklılık gözlemlenmiştir. (p<0,05). Bu farklılığa cinsel olarak aktif olanlar sebep olmaktadır. (9,42±0,50) ortalamayla cinsel olarak aktif olmayan ve belirtmek istemeyenlerden daha yüksek bilgiye sahip oldukları saptanmıştır.
HPV’yi duyanlar için Mann-WhitneyU test sonuçları incelendiğinde 1. Alt boyut olan HPV bilgisi, 3. Alt boyut olan HPV aşı bilgisi ve toplam skor gruplarında istatistiksel olarak anlamlı bir farklılık gözlemlenmiştir. (p<0,05). HPV’yi duymuş olanların HPV’yi duymamış olanlara göre (15,98±0,67'ye karşı 12,71±2,28) daha yüksek HPV bilgisine sahip olduğu saptanmıştır.

HPV’nin nereden duyulduğu incelenmek istediğinde iste 2. Alt boyut olan HPV aşı programı ve 4. Alt boyut olan HPV tarama testi bakımından gruplar arası farklılık istatistiksel olarak anlamlı çıkmıştır. (p<0,05) Bu farklılığa sebep olan grubun Medya (televizyon, gazete, dergi) olduğu saptanmışıtr. 7,87±0,82 ortalaması ile diğer gruplardan daha düşük bir bilgiye sahiptir.
HPV testini duyanlar için Mann-WhitneyU test sonuçları incelendiğinde 1. Alt boyut olan HPV bilgisi ve toplam skor gruplarında istatistiksel olarak anlamlı bir farklılık gözlemlenmiştir. (p<0,05). HPV aşısını duymuş olanların HPV aşısını duymamış olanlara göre (16,25±0,36'ya karşı 13,54±1,37) daha yüksek HPV bilgisine sahip olduğu saptanmıştır.
HPV aşısını duyanlar için Mann-WhitneyU test sonuçları incelendiğinde 1. Alt boyut olan HPV bilgisi, 3. Alt boyut olan HPV aşı bilgisi ve toplam skor gruplarında istatistiksel olarak anlamlı bir farklılık gözlemlenmiştir. (p<0,05). HPV aşısını duymuş olanların HPV aşısını duymamış olanlara göre (16,27±0,72'ye karşı 13,18±1,34) daha yüksek HPV bilgisine sahip olduğu saptanmıştır.

HPV aşısı olanlar için Mann-WhitneyU test sonuçları incelendiğinde 2. Alt boyut olan HPV aşı programı bilgisi ve 3. Alt boyut olan HPV aşı bilgisi gruplarında istatistiksel olarak anlamlı bir farklılık gözlemlenmiştir. (p<0,05). HPV aşısını duymuş olanların HPV aşısını duymamış olanlara göre daha yüksek HPV bilgisine sahip olduğu saptanmıştır.
Henüz HPV aşısı olmamış olan katılımcıların olmama nedeni incelendiğinde tüm alt boyutlarda farklılık gözlemlenmiştir. (p<0,05) Bu farklılığa sebep olan gruplar Post-Hoc karşılaştırma testleri incelendiğinde “Aşının maaliyeti benim için çok yüksek” cevabını verenler olduğu saptanmıştır. 17,93±1,05 ortalama ile en yüksek bilgi düzeyine sahip olanların HPV ile ilgili bilgi ve tutumlarının yüksek olup aşı maaliyeti yüzünden bireylerin aşıyı yaptıramadığı ortaya çıkmıştır.
Gelecekte HPV aşısı olmak isteyenlerin sonucu incelendiğinde 1. Alt boyut olan HPV genel bilgi, 2. Alt boyut olan HPV aşı programı bilgisi, 3.Alt boyut olan HPV aşı bilgisi ve toplam skor için farklılık saptanmıştır. (p <0,05). Farklılık yaratan grubun aşı olmak istemeyenler olduğu ortaya çıkmıştır. 13,74±1,05 ortalama ile HPV aşısı olan ve olmak isteyenlerden daha düşük bilgiye sahip oldukları saptanmıştır.


\newpage
\begin{center}
    
    \centering
    \includegraphics[width=0.9\linewidth]{Imgs/tablo.png}
    \caption{}
    \label{}
\end{center}


\newpage
\begin{center}
    
    \centering
    \includegraphics[width=0.9\linewidth]{Imgs/tablo2.png}
    \caption{}
    \label{}
\end{center}


\newpage
\begin{center}
    
    \centering
    \includegraphics[width=0.9\linewidth]{Imgs/tablo3.png}
    \caption{}
    \label{}
\end{center}
\newpage
\begin{center}
    
    \centering
    \includegraphics[width=0.9\linewidth]{Imgs/tablo4.png}
    \caption{}
    \label{}
\end{center}
 $^aMann-WhitneyU testi$
 \\
 $^bKruskall Wallis testi$
 \\
\textit{Sütunlardaki farklı harfler gruplar arası anlamlı farklılığı göstermektedir. Genel HPV bilgisini içeren sorular 1. alt boyutu, HPV testi bilgisini içeren sorular 2. alt boyutu, HPV aşı bilgisini içeren sorular 3. alt boyutu ve aşılama programına yönelik sorular 4. alt boyutu oluşturmaktadır.}
\newpage
\section{SONUÇLAR ve TARTIŞMA}
\subsection{Sonuçlar}
Bu çalışma, Hacettepe Üniversitesi Beytepe Kampüsü’nde eğitim gören 374 öğrenci arasında gerçekleştirilmiş ve HPV hakkında bilgi düzeyleri, tutumları ve aşılanma durumları incelenmiştir.
Bölgesel köken açısından, İç Anadolu ve Marmara bölgeleri öğrenci popülasyonunun yaklaşık yarısını oluştururken, diğer bölgelerden gelen öğrencilerin oranı daha düşüktür.Kadınların HPV hakkındaki bilgi ve tutumlarının erkeklerden daha yüksek olduğunu göstermiştir. Özellikle HPV genel bilgi ve aşı testi ile ilgili bilgi ve tutumlarda bu fark daha belirgindir. Ayrıca, yaşadıkları bölgeye göre öğrencilerin HPV bilgisi ve tarama testleri bilgisi arasında istatistiksel farklılıklar saptanmıştır. Karadeniz ve Güneydoğu Anadolu Bölgesi’nde yaşayan öğrencilerin bu konulardaki bilgileri diğer bölgelere göre daha düşüktür. Baba eğitim düzeyi de HPV aşı programı bilgisi ve tutumlarında farklılık göstermektedir. İlkokul ve ortaokul mezunu babalara sahip öğrencilerin bu konudaki bilgileri daha düşüktür. Ayrıca, aile gelir durumuna göre HPV genel bilgisi, aşı programı bilgisi ve tarama testi bilgisi arasında farklılıklar gözlemlenmiştir. Daha düşük gelir grubuna sahip öğrencilerin bu konulardaki bilgileri daha düşüktür. Sağlık sigortası durumuna göre de benzer bir eğilim görülmüştür: Sağlık sigortası olmayan öğrencilerin HPV genel bilgisi daha düşüktür. Cinsel aktivite durumuna göre de HPV bilgisi ve aşı bilgisi arasında farklılıklar vardır; cinsel aktif olan öğrencilerin bu konulardaki bilgileri daha yüksektir
Katılımcıların \%89,8’i HPV hakkında bilgi sahibi olup, bu bilgiyi en çok eğitim kurumları ve sosyal medya aracılığıyla edinmişlerdir. HPV testi ve aşısı hakkında bilgi sahibi olanların oranı sırasıyla \%77,8 ve \%79,8’dir. Ancak, HPV aşısı olanların oranı \%20,6 ile sınırlıdır ve aşı olmama nedenleri arasında maliyet ve risk altında hissetmeme algısı ön plana çıkmaktadır.
\subsection{Tartışma}
Kadın öğrencilerin HPV hakkında daha fazla bilgi sahibi olmaları ve gelecekte aşı olmayı daha fazla düşünmeleri, cinsiyet temelli sağlık eğitim ve farkındalık programlarının etkinliğini göstermektedir. Ebeveynlerin eğitim seviyesinin yüksek olması, öğrencilerin HPV hakkında daha fazla bilgi sahibi olmalarıyla ilişkilendirilmiştir, bu da ailevi eğitim seviyesinin sağlık bilgisi üzerindeki etkisini vurgulamaktadır.
Bölgesel kökenin HPV bilgi ve tutumları üzerinde etkili olduğu görülmüştür. Güneydoğu Anadolu ve Karadeniz Bölgesi’nden gelen öğrencilerin diğer bölgelere kıyasla daha düşük bilgi ve tutum puanlarına sahip olmaları, bölgesel sağlık eğitim ve hizmetlerindeki eşitsizlikleri yansıtmaktadır. Bu durum, bölgesel sağlık eğitim programlarının güçlendirilmesi gerektiğini göstermektedir.
Aile gelir durumu ve sağlık sigortası varlığı gibi ekonomik faktörlerin HPV hakkında bilgi ve tutumları etkilediği gözlemlenmiştir. Düşük gelir grupları ve sağlık sigortası olmayanlar genellikle daha düşük bilgi seviyelerine sahiptir, bu da ekonomik engellerin sağlık bilgisi ve erişimini nasıl kısıtlayabileceğini göstermektedir.
Cinsel aktivitenin HPV bilgisi ve aşı bilgisi ile ilişkili olduğu bulunmuştur. Cinsel olarak aktif bireyler, aktif olmayanlara göre daha yüksek bilgi seviyelerine sahiptir, bu da cinsel sağlık eğitiminin önemini vurgulamaktadır.
HPV hakkında bilgi sahibi olmanın ve HPV testi ve aşısını duymanın, genel HPV bilgisi ve tutumları üzerinde olumlu bir etkisi olduğu saptanmıştır. Bu bulgular, HPV hakkında farkındalığı artırmak ve aşıya erişimi genişletmek için hedeflenmiş stratejiler geliştirmenin önemini vurgulamaktadır. Özellikle, HPV aşısının maliyeti, aşıya erişimi sınırlayan önemli bir faktördür ve bu durum, aşıya erişimde eşitlik sağlamak için politika yapıcıların dikkate alması gereken bir husustur.
Sonuç olarak, bu çalışma, HPV hakkında farkındalık ve bilgi düzeylerinin cinsiyet, eğitim, bölge, ekonomik durum ve cinsel aktivite gibi çeşitli faktörlerle ilişkili olduğunu göstermektedir. Çalışmanın bulguları, HPV hakkında farkındalığı artırmak ve aşıya erişimi genişletmek için hedeflenmiş stratejiler geliştirmenin önemini vurgulamaktadır. Bu stratejiler, özellikle düşük gelirli ve sağlık sigortası olmayan bireyler ile bölgesel eğitim ve sağlık hizmetlerinde eşitsizliklerin giderilmesine odaklanmalıdır. Ayrıca, cinsel sağlık eğitiminin gençler arasında HPV bilgisini artırmada kritik bir rol oynadığı ve bu alanda yapılan yatırımların önemini göstermektedir.
\newpage
\twocolumn
\section{EK-1 ANKET FORMU}
\subsection{Sorular}
\\1-Ankete katılmaya rızanız var mı ?
  \\--Evet & --Hayır

\\2-Cinsiyet
   \\--Erkek &--Kadın

\\3-Doğum yılınızı giriniz


\\4-Fakülte
\\   --Ankara Devlet Konservatuvarı
  \\ --Diş Hekimliği Fakültesi
\\   --Edebiyat Fakültesi
  \\ --Eğitim Fakültesi
  \\ --Fen Fakültesi
  \\ --Güzel Sanatlar Fakültesi
   \\--Hukuk Fakültesi
 \\  --İktisadi ve İdari Bilimler Fakültesi
  \\ --İletişim Fakültesi
  \\ --Mühendislik Fakültesi
   \\--Sağlık Bilimleri Fakültesi
  \\ --Spor Bilimleri Fakültesi
   \\--Tıp Fakültesi

\\5-Eğitim Döneminde İkamet Edilen Yer
  \\ --Aile Evi &--Devlet Yurdu
  \\ --Öğrenci Evi& -- Özel Yurt
  \\ --Diğer

\\6-Hayatınızın En Uzun Dönemini Geçirdiğiniz Bölge
  \\ --Akdeniz
  \\ --Doğu Anadolu
  \\ --Ege
 \\  --Güneydoğu Anadolu
  \\ --İç Anadolu
  \\ --Karadeniz
  \\ --Marmara

\\7-Anne Eğitim Durumu
  \\ --Okuryazar Değil
  \\ --İlkokul
  \\ --Ortaokul
  \\ --Lise
  \\ --Üniversite
   \\--Yüksek Lisans/Doktora

\\8-Baba Eğitim Durumu
  \\ --Okuryazar Değil
  \\ --İlkokul
   \\--Ortaokul
   \\--Lise
   \\--Üniversite
   \\--Yüksek Lisans/Doktora

\\9-Aylık Gelir Durumunuz
  \\ --17.000 TL altı
   \\--17.000-34.000 TL
   \\--34.001-61.000 TL
   \\--61.001-100.000 TL
  \\ --100.001-130.000 TL
   \\--130.000 TL üstü

\\10-Sağlık Sigortası Durumu
  \\ --Sağlık sigortam yok
 \\  --Devlet Sağlık Sigortası
  \\ --Özel Sağlık Sigortası
  \\ --Hem devlet hem özel sağlık sigortam var

\\11-Cinsel olarak aktif misiniz?
  \\ --Evet
  \\ --Hayır
  \\ --Belirtmek istemiyorum
   
\\12-HPV(İnsan Papilloma Virüsü)’yi duymuş muydunuz?
  \\ --Evet
  \\ --Hayır
\\Cevabınız “Evet” ise, HPV’yi nereden duydunuz?
  \\ --Aile
\\   --Bilimsel Makale/Araştırma Dergileri
  \\ --Eğitim Kurumu (dersler, seminerler vs.)
  \\ --Medya (televizyon, gazete, dergi)
  \\ --Sağlık Profesyonelleri (doktor, hemşire, sağlık danışmanı vs.)
  \\ --Sosyal Çevre
  \\ --Sosyal Medya 
  \\ --Diğer

\\14- HPV testini duymuş muydunuz?
  \\ --Evet
   \\--Hayır

\\15- HPV aşısını duymuş muydunuz?
  \\ --Evet 
   \\--Hayır
  
\\16-HPV aşısı oldunuz mu?
  \\ --Evet
   \\--Hayır
\\Cevabınız ”Hayır” ise, HPV aşısı yaptırmamanızın nedeni nedir?
  \\ --Aşı hakkında yeterli bilgiye sahip değilim
   \\--Aşının maliyeti benim için yüksek
  \\ --Aşının etkinliği ve güvenilirliği konusunda şüphelerim var
   \\--Kendimi risk altında hissetmiyorum

\\17-Gelecekte HPV aşısı olmayı düşünüyor musunuz?
  \\ --Evet
   \\--Hayır
   \\--HPV aşısı oldum.

\newpage
\subsection{Ölçek}

\begin{longtable}{|m{10cm}|c|c|c|}
    \caption{HPV ile ilgili ifadeler ve cevap seçenekleri} \label{tab:HPV} \\
    \hline
    & \textbf{Evet} & \textbf{Hayır} & \textbf{Bilmiyorum} \\
    \hline
    \endfirsthead
    \hline
    & \textbf{Evet} & \textbf{Hayır} & \textbf{Bilmiyorum} \\
    \hline
    \endhead
    \hline
    \endfoot
    \hline
    \endlastfoot
    HPV, rahim ağzı kanserine neden olabilir. & & & \\ \hline
    Bir kişi, kendisinde HPV olduğunu bilmeden, yıllarca yaşayabilir. & & & \\ \hline
    Birden fazla cinsel eşe sahip olmak, HPV bulaşma riskini arttırır. & & & \\ \hline
    HPV çok nadir görülür. & & & \\ \hline
    HPV cinsel ilişki sırasında bulaşabilir. & & & \\ \hline
    HPV’nin her zaman gözle görülür belirti ve bulguları vardır. & & & \\ \hline
    Prezervatif kullanmak HPV bulaşma riskini azaltır. & & & \\ \hline
    HPV, HIV/AIDS’e neden olabilir. & & & \\ \hline
    HPV cinsel bölgedeki ciltten- cilde, temas ile bulaşabilir. & & & \\ \hline
    HPV erkeklere bulaşmaz. & & & \\ \hline
    Erken yaşta cinsel ilişkiye girmek, HPV bulaşma riskini arttırır. & & & \\ \hline
    HPV’nin birçok tipi vardır. & & & \\ \hline
    HPV cinsel bölgede siğillere neden olabilir. & & & \\ \hline
    HPV antibiyotiklerle tedavi edilebilir. & & & \\ \hline
    Cinsel açıdan aktif olan kişilerin çoğuna, yaşamlarının bir döneminde HPV bulaşacaktır. & & & \\ \hline
    HPV’de genellikle herhangi bir tedaviye gerek yoktur. & & & \\ \hline
    HPV testi size ne kadar zamandan beridir, HPV enfeksiyonunuz olduğunu söyler. & & & \\ \hline
    HPV testi, HPV aşısının gerekli olup olmadığını belirlemek için kullanılır. & & & \\ \hline
    HPV testi yaptırdığınız zaman sonuçlarınızı aynı gün içinde alabilirsiniz. & & & \\ \hline
    HPV testi bir kadında HPV olmadığını gösteriyorsa, o kadının rahim ağzı kanserine yakalanma riski düşüktür. & & & \\ \hline
    HPV aşısı olan kızların ileri yaşlarında smear testi yaptırmasına gerek yoktur. & & & \\ \hline
    HPV aşılarından birisi cinsel bölgedeki siğillere karşı koruma sağlar. & & & \\ \hline
    HPV aşıları cinsel yolla bulaşan tüm enfeksiyonlara karşı koruma sağlar. & & & \\ \hline
    HPV aşısı yapılmış olan bir kişi rahim ağzı kanserine yakalanmaz. & & & \\ \hline
    HPV aşıları, rahim ağzı kanser türlerinin birçoğundan korur. & & & \\ \hline
    HPV aşısının üç doz yapılması gerekir. & & & \\ \hline
    HPV aşılarının en etkili olduğu bireyler hiç cinsel ilişkide bulunmamış olanlardır. & & & \\ \hline
    HPV aşısı 11-26 yaşlar arasındaki tüm kadınlara önerilir. & & & \\ \hline
    HPV aşısı 30-45 yaşlarındaki kadınlar için lisanslıdır (ruhsatlıdır-izinlidir). & & & \\ \hline
    Mevcut olan her iki HPV aşısı da (Gardasil ve Cervarix) hem cinsel bölge siğillerine hem de rahim ağzı kanserine karşı koruma sağlar. & & & \\ \hline
    HPV aşısının 11-26 yaşlar arasındaki erkeklere yapılmasına izin verilmiştir. & & & \\ \hline
\end{longtable}









\clearpage
\begin{onecolumn}
\section{KAYNAKÇA}
    
\begin{enumerate}
   
    \item \href{https://www.hacettepe.edu.tr/ogretim/sayilarla_ogretim}{Hacettepe Üniversitesi Fakülte Bazlı Numerik Genel Bilgiler}
    \item\href{https://tez.yok.gov.tr/UlusalTezMerkezi/tezDetay.jsp?id=PbXEf_W5BPYbJWlM7wlVgQ&no=3YBI6ZjejEkzQxqlVU-UQw}{Demir F(2019)Human Papilloma Virüsü (HPV) Bilgi Ölçeği’nin Türkçe Geçerlik ve Güvenirliği}
    \item \href{https://doi.org/10.29058/mjwbs.974567}{Turhan Çakır A, Porsuk İ, Çalbıyık F, Taner G, Noğay AE, Aslan ME, Demir S, Can C, Altıner E, Kılıç ŞŞ, Yılmaz Ş, Karaca Z, Gündoğan Y, Şen A. Üniversite Öğrencilerinin HPV, HPV Tarama Testi ve HPV Aşısına İlişkin Bilgi Düzeylerinin Değerlendirilmesi: Kesitsel Bir Çalışma. Med J West Black Sea. 2021;5(3):472-80. }
    \item \href{https://doi.org/10.24898/tandro.2023.75875}{Esencan, T.Y,  Yıldırım, A.D. ve Yıldız M. (2023).Human papilloma virüsü farkındalık ve endişe düzeyi ölçeği (HPV-FEÖ): Ölçek geliştirme çalışması. Androloji Bulteni, 25(4),239−245.}
    \item \href{https://doi.org/10.54308/tahd.2023.96630}{Yılmaz Özdemir R, Marakoğlu K, Körez MK. Tıp fakültesi öğrencilerinin human papillomavirüs ve human papillomavirüs aşısı hakkında bilgi, tutum ve davranışlarının değerlendirilmesi. Türk Aile Hek Derg. 2023;27(4):88-94.}
    \item \href{https://doi.org/10.52538/iduhes.1021327}{Güllü, F. N., & Tümer, A. (2022). Türkiye’de Son 10 Yılda Hemşirelik Alanında Yapılan Human Papilloma Vırus Konulu Makalelerin İncelenmesi. Izmir Democracy University Health Sciences Journal, 5(1), 72-86. }
     \item \href{https://hsgm.saglik.gov.tr/depo/birimler/kanser-db/Dokumanlar/Istatistikler/Kanser_Rapor_2018.pdf#:~:text=URL%3A%20https%3A%2F%2Fhsgm.saglik.gov.tr%2Fdepo%2Fbirimler%2Fkanser}{Türkiye Kanser İstatistikleri (2018)}
    \item \href{https://www.klimik.org.tr/wp-content/uploads/2023/04/Emre.Koc_-Uydu-Sunumu_KLIMIK-14.03.2023_compressed.pdf}{HPV İlişkili Kanserler ve Hastalıklar, HPV Aşılar}
    \item \href{https://docs.google.com/forms/d/e/1FAIpQLSdxOtLE8ORo710bd4GC0ps63wlUJTkin91RevpusbA8a_pK_Q/viewform}{Form Bağlantısı} \label{form}
\end{enumerate}
\end{onecolumn}
\end{document}
